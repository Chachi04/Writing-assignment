%%%%%%%%%%%%%%%%%%%%%%%%%%%%% Define Article %%%%%%%%%%%%%%%%%%%%%%%%%%%%%%%%%%
\documentclass{article}
%%%%%%%%%%%%%%%%%%%%%%%%%%%%%%%%%%%%%%%%%%%%%%%%%%%%%%%%%%%%%%%%%%%%%%%%%%%%%%%

%%%%%%%%%%%%%%%%%%%%%%%%%%%%% Using Packages %%%%%%%%%%%%%%%%%%%%%%%%%%%%%%%%%%
\usepackage{geometry}
% \usepackage{amssymb}
\usepackage{amsmath}
\usepackage{amsthm}
\usepackage{pgfplots}
%%%%%%%%%%%%%%%%%%%%%%%%%%%%%%%%%%%%%%%%%%%%%%%%%%%%%%%%%%%%%%%%%%%%%%%%%%%%%%%

% Other Settings
\usepackage[absolute]{textpos}
\usetikzlibrary{calc}

%%%%%%%%%%%%%%%%%%%%%%%%%% Page Setting %%%%%%%%%%%%%%%%%%%%%%%%%%%%%%%%%%%%%%%
\geometry{a4paper}


%%%%%%%%%%%%%%%%%%%%%%%%%% Define vector command %%%%%%%%%%%%%%%%%%%%%%%%%%%%%%%%%%%%%%%
\renewcommand{\v}[1]{\underline{#1}}

\begin{document}
    % \maketitle
    \begin{titlepage}
    \begin{center}
        \includegraphics[height=2cm]{figures/tue-logo-high} \\

        \large
        Department of Mathematics and Computer Science  \\
        Course code: 2MBA20 \\
        Course name: Linear Algebra 1 \\

        \vspace*{5cm}

        \huge
        \textbf{Lines and Ratios in a Triangle } \\
        \Large
        \vspace*{10mm}
        Jiaqi Wang (1986619) \\
        Mil Majerus  \\
        Jean Nguyen \\
        Long Pham \\

        \vfill
        \today
    \end{center}
\end{titlepage}
    \newpage
    \section{Problem description}
    % \section{Somthing}
    In $ \triangle ABC$ (the points $A,B,C$ are non-colinear)
    $P$ is the midpoint of the segment $BC$ and
    $R$ is the point on the line $AB$ such that $A$
    is the midpoint of the segment $BR$. Use vectors to determine the point of intersection
    $Q$ of lines $PR$ and $AC$, and show that $AQ:QC = 1:2$.

    \begin{figure}[h]
        \begin{center}
            \begin{tikzpicture}
                \def\A{$A$}
                \def\B{$B$}
                \def\C{$C$}
                \def\P{$P$}
                \def\Q{$Q$}
                \def\R{$R$}
                \coordinate[label=below:\A] (A) at (0,0) ;
                \coordinate[label=right:\B] (B) at (5,0) ;
                \coordinate[label=above:\C] (C) at (1,4) ;
                \coordinate[label=right:\P] (P) at ($(B)!0.5!(C)$);
                \coordinate[label=left:\R] (R) at (-5,0);
                \coordinate[label=above:\Q] (Q) at ($(A)!1/3!(C)$);
                \draw (A) -- (B) -- (C) -- cycle;
                \draw[dashed] (R) -- (A) ;
                \draw[dashed] (R) -- (P) ;
                \node[circle, fill=blue, inner sep=1pt] at (A) {};
                \node[circle, fill=blue, inner sep=1pt] at (B) {};
                \node[circle, fill=blue, inner sep=1pt] at (C) {};
                \node[circle, fill=blue, inner sep=1pt] at (P) {};
                \node[circle, fill=blue, inner sep=1pt] at (R) {};
                \node[circle, fill=blue, inner sep=1pt] at (Q) {};
            \end{tikzpicture}
        \end{center}
        \caption{Example contructions}
    \end{figure}

    \section{Solution}
    In solving the earlier described problem we will mainly use vector techniques.
    \par
    In geometric exercises that envolve vectors we need to first choose where to put the origin
    if it's not explicitly specified. Without loss of generality, we can choose point $A$ as our origin and assign vectors
    $\v{b}, \v{c}, \v{p}, \v{q}, \v{r}, \v{q}$ to the points
    $B, C, P, Q, R, Q$ respectively.

    Since $A$ is the midpoint of the segment $RB$ and the origin, we get $\v{r} = - \v{b}$. Additionally, since $P$ is the
    midpoint of the segment $BC$, we get $\v{p} = \v{b} + \frac{1}{2}\left(\v{c} - \v{b}\right) = \frac{1}{2} \v{b} + \frac{1}{2} \v{c}$.
    \par
    The line $RP$ can be defined as $RP: \v{x} = \v{r} + \lambda \left(\v{p} - \v{r}\right)$ \cite{vecLineRep}, using our above defined substitutions we thus get
    $RP: \v{x} = - \v{b} + \lambda \left(\frac{3}{2} \v{b} + \frac{1}{2} \v{c}\right)$. We then find the intersection of $RP$ and $AC: \v{x} = \mu \v{c}$.

    \begin{align*}
        - \v{b} + \lambda \left(\frac{3}{2} \v{b} + \frac{1}{2} \v{c}\right) &= \mu \v{c} \\
        - \v{b} + \lambda \left(\frac{3}{2} \v{b} + \frac{1}{2} \v{c}\right) - \mu \v{c} &= \v{0} \\
        \left(-1 + \frac{3}{2} \lambda\right) \v{b} + \left(\frac{1}{2}\lambda - \mu\right)\v{c} &= \v{0} \\
    \end{align*}

    Since $\v{b}$ and $\v{c}$ are linearly independant \cite{linIndepVecs} we get
    \begin{align*}
        &\begin{cases}
            -1 + \frac{3}{2} \lambda = 0 \\
            \frac{1}{2}\lambda - \mu = 0 \\
        \end{cases} \\
        \iff
        &\begin{cases}
            \frac{3}{2} \lambda = 1 \\
            \mu = \frac{1}{2} \lambda \\
        \end{cases} \\
        \iff
        &\begin{cases}
            \lambda = \frac{2}{3} \\
            \mu = \frac{1}{3} \\
        \end{cases}
    \end{align*}
    Since $\v{q} \in AC$, we can write $\v{q} = \mu \v{c} = \frac{1}{3} \v{c}$.
    From this we get $\lvert AQ \rvert = \lVert\frac{1}{3} \v{c}\rVert = \frac{1}{3}\lVert\v{c}\rVert$
    and $\lvert QC \rvert = \lvert AC - AQ\rvert = 72\lVert\v{c} - \frac{1}{3} \v{c}\rVert = \lVert\frac{2}{3} \v{c}\rVert = \frac{2}{3}\lVert\v{c}\rVert$.
    Having calculated the lengths of $AQ$ and $QC$ we see that $\lvert QC \rvert = 2 \cdot \lvert AQ \rvert$.

    \section{Conclusion}
    Indeed, we find that the intersection $Q$ of lines $RP$ and $AC$ divides the segment $AC$
    into two segments $AQ$ and $QC$ where $QC$ is twice the length of $AQ$ (i.e. $AQ:QC = 1:2$).

    \subsection*{Remark}
    Of course there are different ways that we could have solved this exercise. For instance, instead of choosing $A$ as origin
    we could have chosen any other given point or even an arbitrary point in the plane. This, however, may lead
    to more complicated expressions for the lines $RP$ and $AC$, thus making it more prone to computational errors.

    \section{Roles of Homework group members}
    \begin{itemize}
        \item Jiaqi Wang - document organization and visual aspects
        \item Mil Majorus - writing the solution
        \item Jean Nguyen
        \item Long Pham
    \end{itemize}

    \bibliographystyle{plain}
    \bibliography{Writing_Assignment}

\end{document}