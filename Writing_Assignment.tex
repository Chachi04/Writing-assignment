%%%%%%%%%%%%%%%%%%%%%%%%%%%%% Define Article %%%%%%%%%%%%%%%%%%%%%%%%%%%%%%%%%%
\documentclass{article}
%%%%%%%%%%%%%%%%%%%%%%%%%%%%%%%%%%%%%%%%%%%%%%%%%%%%%%%%%%%%%%%%%%%%%%%%%%%%%%%

%%%%%%%%%%%%%%%%%%%%%%%%%%%%% Using Packages %%%%%%%%%%%%%%%%%%%%%%%%%%%%%%%%%%
\usepackage{geometry}
\usepackage{graphicx}
\usepackage{amssymb}
\usepackage{amsmath}
\usepackage{amsthm}
\usepackage{empheq}
\usepackage{mdframed}
\usepackage{booktabs}
\usepackage{graphicx}
\usepackage{color}
\usepackage{psfrag}
\usepackage{pgfplots}
\usepackage{bm}
%%%%%%%%%%%%%%%%%%%%%%%%%%%%%%%%%%%%%%%%%%%%%%%%%%%%%%%%%%%%%%%%%%%%%%%%%%%%%%%

% Other Settings
\usetikzlibrary{calc}

%%%%%%%%%%%%%%%%%%%%%%%%%% Page Setting %%%%%%%%%%%%%%%%%%%%%%%%%%%%%%%%%%%%%%%
\geometry{a4paper}

%%%%%%%%%%%%%%%%%%%%%%%%%% Define some useful colors %%%%%%%%%%%%%%%%%%%%%%%%%%
\definecolor{ocre}{RGB}{243,102,25}
\definecolor{mygray}{RGB}{243,243,244}
\definecolor{deepGreen}{RGB}{26,111,0}
\definecolor{shallowGreen}{RGB}{235,255,255}
\definecolor{deepBlue}{RGB}{61,124,222}
\definecolor{shallowBlue}{RGB}{235,249,255}
%%%%%%%%%%%%%%%%%%%%%%%%%%%%%%%%%%%%%%%%%%%%%%%%%%%%%%%%%%%%%%%%%%%%%%%%%%%%%%%

%%%%%%%%%%%%%%%%%%%%%%%%%% Define an orangebox command %%%%%%%%%%%%%%%%%%%%%%%%
\newcommand\orangebox[1]{\fcolorbox{ocre}{mygray}{\hspace{1em}#1\hspace{1em}}}
\renewcommand{\v}[1]{\underbar{#1}}
%%%%%%%%%%%%%%%%%%%%%%%%%%%%%%%%%%%%%%%%%%%%%%%%%%%%%%%%%%%%%%%%%%%%%%%%%%%%%%%

%%%%%%%%%%%%%%%%%%%%%%%%%%%% English Environments %%%%%%%%%%%%%%%%%%%%%%%%%%%%%
\newtheoremstyle{mytheoremstyle}{3pt}{3pt}{\normalfont}{0cm}{\rmfamily\bfseries}{}{1em}{{\color{black}\thmname{#1}~\thmnumber{#2}}\thmnote{\,--\,#3}}
\newtheoremstyle{myproblemstyle}{3pt}{3pt}{\normalfont}{0cm}{\rmfamily\bfseries}{}{1em}{{\color{black}\thmname{#1}~\thmnumber{#2}}\thmnote{\,--\,#3}}
\theoremstyle{mytheoremstyle}
\newmdtheoremenv[linewidth=1pt,backgroundcolor=shallowGreen,linecolor=deepGreen,leftmargin=0pt,innerleftmargin=20pt,innerrightmargin=20pt,]{theorem}{Theorem}[section]
\theoremstyle{mytheoremstyle}
\newmdtheoremenv[linewidth=1pt,backgroundcolor=shallowBlue,linecolor=deepBlue,leftmargin=0pt,innerleftmargin=20pt,innerrightmargin=20pt,]{definition}{Definition}[section]
\theoremstyle{myproblemstyle}
\newmdtheoremenv[linecolor=black,leftmargin=0pt,innerleftmargin=10pt,innerrightmargin=10pt,]{problem}{Problem}[section]
%%%%%%%%%%%%%%%%%%%%%%%%%%%%%%%%%%%%%%%%%%%%%%%%%%%%%%%%%%%%%%%%%%%%%%%%%%%%%%%

%%%%%%%%%%%%%%%%%%%%%%%%%%%%%%% Plotting Settings %%%%%%%%%%%%%%%%%%%%%%%%%%%%%
\usepgfplotslibrary{colorbrewer}
\pgfplotsset{width=8cm,compat=1.9}
%%%%%%%%%%%%%%%%%%%%%%%%%%%%%%%%%%%%%%%%%%%%%%%%%%%%%%%%%%%%%%%%%%%%%%%%%%%%%%%

%%%%%%%%%%%%%%%%%%%%%%%%%%%%%%% Title & Author %%%%%%%%%%%%%%%%%%%%%%%%%%%%%%%%
\title{Lines and ratios in a triangle}
\author{Jiaqi Wang \\ 1986619
\and Mil
\and Jean
\and Tom}
%%%%%%%%%%%%%%%%%%%%%%%%%%%%%%%%%%%%%%%%%%%%%%%%%%%%%%%%%%%%%%%%%%%%%%%%%%%%%%%

\begin{document}
    \maketitle
    \newpage
    \section{Problem description}
    % \section{Somthing}
    In $ \triangle ABC$ (the points $A,B,C$ are non-colinear)
    $P$ is the midpoint of the segment $BC$ and
    $R$ is the point on the line $AB$ such that $A$
    is the midpoint of the segment $BR$. Use vectors to determine the point of intersection
    $Q$ of lines $PR$ and $AC$, and show that $AQ:QC = 1:2$.

    \begin{figure}[h]
        \begin{center}
            \begin{tikzpicture}
                \def\A{$A$}
                \def\B{$B$}
                \def\C{$C$}
                \def\P{$P$}
                \def\Q{$Q$}
                \def\R{$R$}
                \coordinate[label=below:\A] (A) at (0,0) ;
                \coordinate[label=right:\B] (B) at (5,0) ;
                \coordinate[label=above:\C] (C) at (2,4) ;
                \coordinate[label=right:\P] (P) at ($(B)!0.5!(C)$);
                \coordinate[label=left:\R] (R) at (-5,0);
                \coordinate[label=above:\Q] (Q) at ($(A)!1/3!(C)$);
                \draw (A) -- (B) -- (C) -- cycle;
                \draw[dashed] (R) -- (A) ;
                \draw[dashed] (R) -- (P) ;
                \node[circle, fill=blue, inner sep=1pt] at (A) {};
                \node[circle, fill=blue, inner sep=1pt] at (B) {};
                \node[circle, fill=blue, inner sep=1pt] at (C) {};
                \node[circle, fill=blue, inner sep=1pt] at (P) {};
                \node[circle, fill=blue, inner sep=1pt] at (R) {};
                \node[circle, fill=blue, inner sep=1pt] at (Q) {};
            \end{tikzpicture}
        \end{center}
        \caption{Example contructions}
    \end{figure}

    
    \section{Solution}
    In solving the earlier described problem we will mainly use vector techniques. 
    \par
    In geometric exercises that envolve vectors we need to first choose where to put the origin
    if it's not explicitly specified. Without loss of generality, we will choose point $A$ as our origin and assign vectors
    $\v{b}, \v{c}, \v{p}, \v{q}, \v{r}, \v{q}$ to the points
    $B, C, P, Q, R, Q$ respectively.

    Since $A$ is the midpoint of the segment $RB$ and the origin, we get $\v{r} = - \v{b}$. Additionally, since $P$ is the 
    midpoint of the segment $BC$, we get $\v{p} = \v{b} + \frac{1}{2}(\v{c} - \v{b}) = \frac{1}{2} \v{b} + \frac{1}{2} \v{c}$. \\
    The line $RP$ can be defined as $RP: \v{x} = \v{r} + \lambda (\v{p} - \v{r})$ \cite{vecLineRep}, using our above defined substitutions we thus get
    $RP: \v{x} = - \v{b} + \lambda (\frac{3}{2} \v{b} + \frac{1}{2} \v{c})$. We then find the intersection of $RP$ and $AC: \v{x} = \mu \v{c}$.
    
    \begin{align*}
        - \v{b} + \lambda (\frac{3}{2} \v{b} + \frac{1}{2} \v{c}) &= \mu \v{c} \\
        - \v{b} + \lambda (\frac{3}{2} \v{b} + \frac{1}{2} \v{c}) - \mu \v{c} &= \v{0} \\
        (-1 + \frac{3}{2} \lambda) \v{b} + (\frac{1}{2}\lambda - \mu)\v{c} &= \v{0} \\
    \end{align*}
    
Since $\v{b}$ and $\v{c}$ are linearly independant \cite{linIndepVecs} we get 
$$
\begin{cases}
    -1 + \frac{3}{2} \lambda = 0 \\
    \frac{1}{2}\lambda - \mu = 0 \\
\end{cases}
\iff
\begin{cases}
    \frac{3}{2} \lambda = 1 \\
    \mu = \frac{1}{2} \lambda \\
\end{cases}
\iff
\begin{cases}
    \lambda = \frac{2}{3} \\
    \mu = \frac{1}{3} \\
\end{cases}
$$
Since $\v{q} \in AC$, we can write $\v{q} = \mu \v{c} = \frac{1}{3} \v{c}$. From this we get $|AQ| = ||\frac{1}{3} \v{c}|| = \frac{1}{3}||\v{c}||$
and $|QC| = |AC - AQ| = 7|\v{c} - \frac{1}{3} \v{c}|| = ||\frac{2}{3} \v{c}|| = \frac{2}{3}||\v{c}||$.
Having calculated the lengths of $AQ$ and $QC$ we see that $|QC| = 2 \cdot |AQ|$.

 
    \section{Conclusion}
    Indeed, we find that the intersection $Q$ of lines $RP$ and $AC$ divides the segment $AC$ 
    into two segments $AQ$ and $QC$ where $QC$ is twice the length of $AQ$ ($AQ:QC = 1:2$).

    \subsection*{Remark}
    Of course there are different ways that we could have solved this exercis

    \section{Role of Homework group members}
    \bibliographystyle{plain}
    \bibliography{Writing_Assignment}    
    
\end{document}